\documentclass[11pt]{article}
\usepackage[utf8]{inputenc}   
\usepackage[T1]{fontenc}
\usepackage[french]{babel}
%\usepackage[latin1]{inputenc}
\usepackage{graphicx}
\usepackage{lmodern}
\usepackage{amssymb,amsmath,latexsym,amsfonts}
\usepackage{color}
\usepackage{eurosym}
\usepackage{fancyhdr}
\usepackage{hyperref}
\pagestyle{fancy}
\lhead{Cahier des charges Technique}
\rhead{Parc informatique}
\hypersetup{
    linkcolor= blue,               % parametrage des hyperliens
    colorlinks=true,                % colorise les liens
    breaklinks=true,                
    urlcolor= blue,                 % couleur des hyperliens
                    % couleur des liens internes aux documents (index, 
                   % couleur des liens vers les references b
    pdftitle={TER}, %informations apparaissant dans
    pdfauthor={HAMRI Nouredine},     %dans les informations du document
    pdfsubject={Mac OS X}          %sous Acrobat.
}


\title{\begin{center}\hspace{1.5cm}Cahier Des Charges Du Projet : \newline
              \newline
               \newline
 \textbf{ Automatisation et décentralisation de l'installation/désinstallation de logiciels dans un parc informatique} 
  \end{center}}


\begin{document}
\maketitle

\newpage

\tableofcontents

\newpage
\section{Présentation du problème}
\subsection{Objectifs, Contraintes et environnement}
Le but du projet est de créer un programme pouvant répartir les installations et désinstallation d'un programme sur l'ensemble des machines composant un parc informatique (voir schéma ci-dessous).

Ce parc informatique est composé de machines identiques sous le système d'exploitation Windows, et pour la réalisation de ce projet, le langage PERL nous est imposé.
\begin{figure}[h!]
    \center
    \includegraphics[scale=0.4]{Diagrammes/png/CasUtilisationGeneral.png}
    \caption{Cas d'utilisation de l'utilisation du logiciel d'administration}
\end{figure}




\section{Solution logicielle}
\subsection{Modélisation des communications}
Les différents types de communications ont été modélisés sur les schémas suivants :

\begin{figure}[h!]
    \center
    \includegraphics[scale=0.2]{Diagrammes/png/Desinstallation.png}
    \caption{Séquence désinstallation d'un programme}
\end{figure}
\begin{figure}[h!]
    \center
    \includegraphics[scale=0.2]{Diagrammes/png/Installation.png}
    \caption{Séquence d'installation d'un programme}
\end{figure}
\begin{figure}[h!]
    \center
    \includegraphics[scale=0.2]{Diagrammes/png/NouvelleMachine.png}
    \caption{Installation d'une nouvelle machine dans le parc}
\end{figure}
\newpage






\subsection{Architecture du système}
Pour la réalisation de ce logiciel, le modèle d'architecture qui sera utilisé sera adapté du modèle MIMD (Multiple Instructions Multiple Data). Au vu des communications réseau à effectuer, le modèle de gestion par message, semble être le plus adapté.
\subsection{Modules utilisées}
Pour ce projet, nous utiliserons les modules suivants :

\begin{itemize}
     \item Win32TieRegistry : Module disponible sur le site de la CPAN permettant d'effectuer les manipulations sur les clés de registre ; 
     \item \textbf{Socket :} Module également disponible sur le site de la CPAN, il permet la mise en place de Socket pour l'interaction en réseau ;
     \item DBD-SQLite : permettant d'établir des bases de données concernant les différents produits installés ;
  \end{itemize}
\end{document}